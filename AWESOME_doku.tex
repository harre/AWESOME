\documentclass[a4paper,english,10.5pt]{scrartcl}
\linespread{1.1}
\usepackage{amsmath, amssymb, latexsym}
%\usepackage{ngerman}
\usepackage[T1]{fontenc}
\usepackage[latin1]{inputenc}
\usepackage{bibentry}
\usepackage{lmodern} 
\usepackage{grffile}
\usepackage{pdfpages}
\usepackage[pdftex]{hyperref}
\usepackage{natbib}
\hypersetup
{%
  pdftitle={AWESOME documentation},
  pdfauthor={Harald Hoeller, Markus Haider},
  pdfsubject={...},
  pdfkeywords={AWEsome Simulations Of Metal Enrichment}
}
\usepackage{lscape}
\usepackage{multicol}
\usepackage{ae}
\usepackage{color}
\usepackage{aecompl}
\usepackage[T1]{fontenc}
\usepackage[latin1]{inputenc}
\pagestyle{headings}
\setcounter{secnumdepth}{3}
\setcounter{tocdepth}{3}
\setlength\parskip{\smallskipamount}
%\setlength\parindent{0pt}
\usepackage{amsmath}
\usepackage{graphicx}
%\usepackage{subfigure}
\usepackage{subfig}
%\usepackage[countmax]{subfloat}
\usepackage{xcolor}
\usepackage{setspace}
\usepackage{appendix}
\usepackage{amssymb}
\usepackage{amsthm}
\usepackage{wrapfig}
\usepackage{bm}
\usepackage{hyperref}
\usepackage{fancybox} 
\usepackage{ifthen} 
\renewcommand{\theenumi}{\emph{\small{\roman{enumi}}.)}}
\renewcommand{\labelenumi}{\theenumi}
\title{AWESOME documentation \\
metal enrichment setup 2011ff}
\author{Harald H\"oller, Markus Haider \thanks{Department of Astro- and Particle Physics, 
University of Innsbruck, harald.hoeller@uibk.ac.at}}

\begin{document}
\maketitle 

\begin{abstract}
\textcolor{red}{\bf FOR INTERNAL AWESOME USE}
\end{abstract}

\section{Initial Conditions}
\subsection{Using Cosmics 1.04 (Standard)}

The (constrained) initial conditions generator \texttt{Cosmics 1.04} 
can be downloaded from \texttt{http://web.mit.edu/edbert/cosmics-1.04.tar.gz}. 
The acually used version is to be gotten as git repository
\texttt{git@github.com:harre/cosmics-initial-conditions.git}. 

\begin{description}

\item[Step 1: Make] 

For making, one has to specify the system and adapt the corresponding 
Makefile accordingly. In the folder \texttt{Make\_files} one has to adapt 
\texttt{Make.LINUX} to 
\begin{verbatim}
F77 = ifort
F77FLAGS = -O2 -parallel -par-report1 -openmp
FFT_OBJ = fft3r.o
CC = icc
CFLAGS = -O2 -parallel -par-report1 -openmp
\end{verbatim}
and load the intel compiler \texttt{module load intel/64/12.1}. The intel 
compiler is used since for $512^3$ particle initial conditions there is a 
problem with memory allocation with \texttt{gcc} (can be dealt with flags however). 
When one uses the version from the git repo, changing the 
Makefile should not be necessary anyhow. 

\item[Step 2: Linear Extrapolation: \texttt{linger}] 

In the folder \texttt{linger\_syn} the program with the same name is to 
be executed. This program generates a file called \texttt{linger.dat} which 
is then used by the actual IC generator \texttt{grafic}. 

When one executes the program from github, the message 
\begin{verbatim} 
 	
 >>>>>    MAKE SURE YOU START THE PROGRAM WITH
 >>>>>    $./linger_syn | tee lingersynIO.out
 	
\end{verbatim}
motivates the recording of IO in order to ensure reproducability of simulations. 
Ideally one would also indicate some physical parameters in 
the IO file, such as \texttt{lingersynIO\_h70.out}. This file is 
consequently copied from further simulation setup scripts (see later 
items and sections). 

\textbf{Note: } There are files already in the repo which can be used, namely 
\texttt{linger\_h100n216.dat} and  \texttt{linger\_h70n216.dat} for 
Hubble constants $70.3$ and $100$. They were 
generated using matter transfer functions (choice 0) 
\begin{verbatim}
 Enter 1 for full Boltzmann equation for CMB (lmax<=10000, linear k)
    or 0 for matter transfer functions only (lmax=100, log k)
\end{verbatim}
and the parameters 
\begin{verbatim}
 Enter kmin (1/Mpc), kmax (1/Mpc), nk, zend
\end{verbatim}
are set to \texttt{1E-5}, \texttt{50}, \texttt{216}, \texttt{0}. 
These numbers are chosen partially arbitrary but have proven 
robust. The upper limit for the wave number certainly will be too small 
for larger volumes than $100$Mpc (rem.: Nyquist frequency). The 
number of $k$ to be calculated are chosen in a way that the calculation 
does not last too long. 

The last input is then which kind of IC one wants to generate 
\begin{verbatim}
 Initial conditions cases:
     1 for isentropic (adiabatic) fluctuations,
     2 for cdm entropy/isocurvature fluctuations, or
     3 for baryon entropy/isocurvature fluctuations, or
     4 for seed/isocurvature fluctuations
 Enter 1, 2, 3, or 4 now
\end{verbatim}
where we chose \texttt{2}. 

\textbf{Note: } If one changes the cosmological parameters for 
\texttt{linger\_syn} one usually hast to delete the files \texttt{linger.dat} 
and \text{lingerg.dat} first. 

\end{description}



\subsection{Using NGenIC}

\section{N-body Simulation}

\section{Halo Finder + Merger Trees}

\section{SAM}
\subsection{Galacticus}
\subsubsection{Compilation}


\subsection{Galacticus v0.9.1}

Checkout latest revision: 
\begin{verbatim}
bzr checkout http://bazaar.launchpad.net/~abenson/galacticus/v0.9.1/
\end{verbatim}

\section{Hydro}

\bibliography{AWESOME}
\renewcommand{\bibsection}{\section{References}}
\setlength{\bibhang}{1.24cm}
\setlength{\parindent}{3cm}
\setlength{\bibsep}{0cm}
\bibliographystyle{dcu}
\setcitestyle{authoryear,round,citesep={;},aysep={,},yysep={;}}
\gdef\harvardand{\&}
\end{document}


